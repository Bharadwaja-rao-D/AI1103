\documentclass[journal,12pt,twocolumn]{IEEEtran}
\usepackage{amsthm}
\allowbreak
\usepackage{setspace}
\usepackage{gensymb}
\singlespacing
\usepackage[cmex10]{amsmath}
\usepackage{caption}
\usepackage{amsthm}
\usepackage{float}
\DeclareUnicodeCharacter{2212}{-}
\usepackage{tikz}
\usepackage{pgfplots}

\usepackage{mathrsfs}
\usepackage{txfonts}
\usepackage{stfloats}
\usepackage{bm}
\usepackage{cite}
\usepackage{cases}
\usepackage{subfig}

\usepackage{longtable}
\usepackage{multirow}

\usepackage{enumitem}
\usepackage{mathtools}
\usepackage{steinmetz}
\usepackage{tikz}
\usepackage{circuitikz}
\usepackage{verbatim}
\usepackage{tfrupee}
\usepackage[breaklinks=true]{hyperref}
\usepackage{graphicx}
\usepackage{tkz-euclide}
\graphicspath{ {./images/} }
\usetikzlibrary{calc,math}
\usepackage{listings}
    \usepackage{color}                                            %%
    \usepackage{array}                                            %%
    \usepackage{longtable}                                        %%
    \usepackage{calc}                                             %%
    \usepackage{multirow}                                         %%
    \usepackage{hhline}                                           %%
    \usepackage{ifthen}                                           %%
    \usepackage{lscape}     
\usepackage{multicol}
\usepackage{chngcntr}

\DeclareMathOperator*{\Res}{Res}

\renewcommand\thesection{\arabic{section}}
\renewcommand\thesubsection{\thesection.\arabic{subsection}}
\renewcommand\thesubsubsection{\thesubsection.\arabic{subsubsection}}

\renewcommand\thesectiondis{\arabic{section}}
\renewcommand\thesubsectiondis{\thesectiondis.\arabic{subsection}}
\renewcommand\thesubsubsectiondis{\thesubsectiondis.\arabic{subsubsection}}


\hyphenation{op-tical net-works semi-conduc-tor}
\def\inputGnumericTable{}                                 %%

\lstset{
%language=C,
frame=single, 
breaklines=true,
columns=fullflexible
}
\begin{document}


\newtheorem{theorem}{Theorem}[section]
\newtheorem{problem}{Problem}
\newtheorem{proposition}{Proposition}[section]
\newtheorem{lemma}{Lemma}[section]
\newtheorem{corollary}[theorem]{Corollary}
\newtheorem{example}{Example}[section]
\newtheorem{definition}[problem]{Definition}

\newcommand{\BEQA}{\begin{eqnarray}}
\newcommand{\EEQA}{\end{eqnarray}}
\newcommand{\define}{\stackrel{\triangle}{=}}
\bibliographystyle{IEEEtran}
\raggedbottom
\setlength{\parindent}{0pt}
\providecommand{\mbf}{\mathbf}
\providecommand{\pr}[1]{\ensuremath{\Pr\left(#1\right)}}
\providecommand{\qfunc}[1]{\ensuremath{Q\left(#1\right)}}
\providecommand{\sbrak}[1]{\ensuremath{{}\left[#1\right]}}
\providecommand{\lsbrak}[1]{\ensuremath{{}\left[#1\right.}}
\providecommand{\rsbrak}[1]{\ensuremath{{}\left.#1\right]}}
\providecommand{\brak}[1]{\ensuremath{\left(#1\right)}}
\providecommand{\lbrak}[1]{\ensuremath{\left(#1\right.}}
\providecommand{\rbrak}[1]{\ensuremath{\left.#1\right)}}
\providecommand{\cbrak}[1]{\ensuremath{\left\{#1\right\}}}
\providecommand{\lcbrak}[1]{\ensuremath{\left\{#1\right.}}
\providecommand{\rcbrak}[1]{\ensuremath{\left.#1\right\}}}
\theoremstyle{remark}
\newtheorem{rem}{Remark}
\newcommand{\sgn}{\mathop{\mathrm{sgn}}}
\providecommand{\abs}[1]{$\left\vert#1\right\vert$}
\providecommand{\res}[1]{\Res\displaylimits_{#1}} 
\providecommand{\norm}[1]{$\left\lVert#1\right\rVert$}
%\providecommand{\norm}[1]{\lVert#1\rVert}
\providecommand{\mtx}[1]{\mathbf{#1}}
\providecommand{\mean}[1]{E$\left[ #1 \right]$}
\providecommand{\fourier}{\overset{\mathcal{F}}{ \rightleftharpoons}}
%\providecommand{\hilbert}{\overset{\mathcal{H}}{ \rightleftharpoons}}
\providecommand{\system}{\overset{\mathcal{H}}{ \longleftrightarrow}}
	%\newcommand{\solution}[2]{\textbf{Solution:}{#1}}
\newcommand{\solution}{\noindent \textbf{Solution: }}
\newcommand{\cosec}{\,\text{cosec}\,}
\providecommand{\dec}[2]{\ensuremath{\overset{#1}{\underset{#2}{\gtrless}}}}
\newcommand{\myvec}[1]{\ensuremath{\begin{pmatrix}#1\end{pmatrix}}}
\newcommand{\mydet}[1]{\ensuremath{\begin{vmatrix}#1\end{vmatrix}}}
\numberwithin{equation}{subsection}
\makeatletter
\@addtoreset{figure}{problem}
\makeatother
\let\StandardTheFigure\thefigure
\let\vec\mathbf
\renewcommand{\thefigure}{\theproblem}
\def\putbox#1#2#3{\makebox[0in][l]{\makebox[#1][l]{}\raisebox{\baselineskip}[0in][0in]{\raisebox{#2}[0in][0in]{#3}}}}
     \def\rightbox#1{\makebox[0in][r]{#1}}
     \def\centbox#1{\makebox[0in]{#1}}
     \def\topbox#1{\raisebox{-\baselineskip}[0in][0in]{#1}}
     \def\midbox#1{\raisebox{-0.5\baselineskip}[0in][0in]{#1}}
\vspace{3cm}
\title{AI1103: Assignment 5}
\author{Damaragidda Bharadwaja Rao - CS20BTECH11012}
\maketitle
\newpage
\bigskip
\renewcommand{\thefigure}{\theenumi}
\renewcommand{\thetable}{\theenumi}
Download all latex-tikz codes from 
\begin{lstlisting}
https://github.com/Bharadwaja-rao-D/AI1103/blob/main/assignment5/assignment5.tex
\end{lstlisting}
\section*{Problem UGC-MATH 2019 Q 105:}
Consider a simple symmetric random walk on integers, where from every state i you move to i-1 and i+1 with the probability half each. Then which of the following are true?
\begin{enumerate}
\item The random walk is aperiodic
\item The random walk is irreducible
\item The random walk is null recurrent 
\item The random walk is positive recurrent
\end{enumerate}
\section*{Solution:}
The simple symmetric random walk is a Markov chain with state space S = $\{i | i \in \mathbb{Z}\}$ and with transition matrix P where, 
\begin{equation}
  P(i,j)=\begin{cases}
  0 , &|i-j|>1\vspace{0.2cm}\\
  p = \dfrac{1}{2} , &j = i+1\vspace{0.2cm}\\
  q =1-p = \dfrac{1}{2} , &j = i-1\\
  \end{cases}
\end{equation}
and $P^n(i,j)$ denotes the probability of being in state j, starting from state i after n steps or transitions. 
\subsection{Aperiodic}
For a Markov chain to be aperiodic there should exist an integer k such that
\begin{align}
P^n(j,j) > 0 \hspace{0.5cm} \forall n \geq k \label{eq:aperoidic}
\end{align} 
to return to same state after n steps, number of forward and backward steps should be same, that is number of steps should be even. 
\begin{align}
P^n(j,j) = 0 \hspace{0.5cm} \forall n \in Odd \label{eq:1}
\end{align}
$\therefore$ Equation \eqref{eq:aperoidic} is not satisfied and Option \textbf{(1)} is \textbf{incorrect}.
\subsection{Irreducible}
For a Markov chain to be \textbf{irreducible} all pairs i,j should communicate with each other.Let us assume that the chain starts from state i and let us assume that it requires m forward and (n-m) backward steps to reach j .Let $i < j$ wlog 
\begin{align}
j - i &= m - (n - m) = 2m - n\\
m &= \frac{(j - i) + n}{2}\\
P^n(i,j) &= \binom{n}{m} p^m q^{n-m}\\
P^n(i,j) &= \binom{n}{\dfrac{(j - i) + n}{2}} p^m q^{n-m}\\
P^n(i,j) &> 0 \hspace{0.3cm}  n = (j-i)+2k \hspace{0.2cm} \forall k \in \mathbb{W}
\end{align}
Here i and j are general , hence all pairs i and j communicate with each other.\\
$\therefore$Option \textbf{(2)} is \textbf{correct}
\subsection{Recurrent}
In a Markov Chain for state i to be \textbf{recurrent} it should satisfy ,
\begin{align}
\lim_{t \to \infty} \sum_{n=1}^{t} P^n(i,i) = \infty \label{eq:Recurrent}
\end{align}
\begin{align}
\lim_{t \to \infty} \sum_{n=1}^{t} P^n(i,i) &= \lim_{t \to \infty}\left(\sum_{k=1}^{t} P^{2k}(i,i) + \sum_{k=1}^{t} P^{2k-1}(i,i)\right)\\
&= \lim_{t \to \infty}\sum_{k=1}^{t} P^{2k}(i,i)\\
P^{2k}(i,i) &= \binom{2k}{k}p^kq^k  = \frac{2k !}{k!k!}p^kq^k \label{eq:2}
\end{align}
By using Stirling approximation to \eqref{eq:2} we get 
\begin{align}
&P^{2k}(i,i) = \dfrac{ \left((2k)^{2k+\frac{1}{2}}\right)\times \exp(-2k)\times(2\pi)^{\frac{1}{2}}}{\left(k^{k+\frac{1}{2}}\times \exp(-k)\right)^2\times
2\pi}p^kq^k\\
&= \dfrac{(4pq)^{2k}}{(k\pi)^{\frac{1}{2}}} = \dfrac{1}{(k\pi)^{\frac{1}{2}}}\\
&\lim_{t \to \infty}\sum_{k=1}^{t} P^{2k}(i,i) = \lim_{t \to \infty}\sum_{k=1}^{t} \dfrac{1}{(k\pi)^{\frac{1}{2}}}
\end{align}
Since $\dfrac{1}{k^\frac{1}{2}}$ is divergent, Equation \eqref{eq:Recurrent} is satisfied\\
$\therefore$ The random walk is \textbf{recurrent}\\
The first-passage time probability is 
\begin{align}
f_{i,j}(n) = \Pr(X_n = j, X_{n-1} \neq j, X_{n-2} \neq j, ... X_1 \neq j | X_0 = i )
\end{align}
The first-passage time $T_{i,j}$ from state i to j has the PMf $f_{i,j}(n)$ and the distribution function
\begin{align}
F_{i,j}(n) = \sum _{k=0}^n f_{i,j}(k) \label{eq:3}
\end{align}
For the Markov Chain to be null recurrent 
\begin{align}
\overline{T_{j,j}} = \infty \label{eq:null recurrent}
\end{align}
and for positive recurrent 
\begin{align}
\overline{T_{j,j}} < \infty
\end{align}
where $\overline{T_{j,j}}$ represents the mean time to enter j starting from j. We can calculate the mean by using the distribution function 
\begin{align}
\overline{T_{j,j}} = 1 + \sum _{k=0}^n \left(1 - F_{j,j}(k)\right)
\end{align}
From \eqref{eq:3} we get \eqref{eq:null recurrent} condition to be satisfied \\
$\therefore$ Option \textbf{(3)} is \textbf{correct}.\\
\textbf{Answer :} option2, option3


\end{document}
